\Chapter{Összefoglalás}
\label{Chap:osszefoglalas}
\iffalse 
Ebben a fejezetben kell összefoglalni a szakdolgozat eredményeit, sajátosságait és a témában való elhelyezkedését. A fejezet címe nem módosítható! Lehet benne több alfejezet is, de nem ajánlott. Min 1 max 4 oldal terjedelemben
\fi

A szakdolgozat során alkalmazott gépi tanulási módszerek várható módon eltérő eredményekkel szolgáltak azonban szinte mindegyikre nézve volt olyan terület ahol kiemelkedően jobban teljesítettek mint az adott problémára kipróbált egyéb eljárások. Azonban a gépi tanulás alkalmazása során nincs kőbe vésve hogy egy adott problémára mi a legjobb megoldás, csupán kiindulási pontot lehet meghatározni a területhez kapcsolódó iránymutatásra és gyakorlatra támaszkodva, végső soron pedig az eredmény és a kiválasztott eljárás a felhasznált adathalmaztól függ. 

A szakdolgozat eredményei jól mutatják hogy a gépi tanulás kiválóan használható nem csak a versenyszerű hanem a mindennapi sport esetében is, alkalmazásával kialakíthatóak olyan funkciók amelyek segítenek az átlagos, hobbi szerűen kerékpározóknak az útjaik megtervezésében, edzéseik kialakításában. Ehhez természetesen alapfeltétel egy nagy mennyiségű, változatos adathalmaz megléte, azonban a kerékpározók körében annyira természetes az útvonalaik rögzítése hogy ez nem jelenthet akadályt.

A kerékpározáshoz kapcsolódó alapvetőbb számítások mint a különböző idő intervallumok meghatározása egyszerű regressziókkal könnyen elvégezhetőek és a viszonylag kis méretű adathalmazon is nagy pontosságot eredményeztek. Az összetettebb feladatok megoldása nem ennyire kézenfekvő, a különböző típusú nehézségi osztályok meghatározására nincs egyértelmű jó megoldás, azonban a szakdolgozat során született eredmények egy használható alapot adnak, melyek később továbbfejlesztésre kerülnek. A módszerek és eredmények fejlesztését magába foglaló projekt túlmutat a szakdolgozat keretein, azonban magába foglalná egy web vagy mobil alapú alkalmazás elkészítését, amely lehető tenné a felhasználói számára a szakdolgozat során kidolgozott megoldások használatát. Egy ilyen alkalmazás a felhasználók kiszolgálásán kívül a megoldások fejlesztésének megkönnyítését is szolgálná, mivel egy növekvő felhasználói bázis könnyen sokszorosára emelhetné a a gépi tanulás során felhasználható adatmennyiséget, ezzel javítva az adathalmaz sokszínűségét, lehetővé téve pontosabb nehézségi csoportok kialakítását nem és korcsoportok szerint. A nehézségi csoportok fejlődése hosszútávon hozzájárulna egy összetettebb edzési javaslatokat készítő rendszer megvalósításához, amely a távolságon és átlagos sebességen felül az út egyéb paramétereire is megbízható ajánlást tudna készíteni, akár konkrét, lokáció alapú útvonalrajzolással.


%. Ugyan a szakdolgozat keretein túlmutat de későbbi projekt során kialakításra kerülhetne egy tényleges webes vagy mobilos alkalmazás, amely lehetővé teszi a felhasználói számára a szakdolgozat során kidolgozott módszerek továbbfejlesztett változatainak a használatát. Egy ilyen rendszer a módszerek fejlesztésének megkönnyítését is szolgálná, mivel egy növekvő felhasználói bázis akár sokszorosára is emelhetné a gépi tanulás során felhasználható adatok mennyiségét, ezzel javítva az adathalmaz sokszínűségét, lehetővé téve pontosabb nehézségi csoportok kialakítását nem és kor szerint. A nehézségi csoportok fejlődése hosszútávon hozzájárulna egy összetettebb edzési javaslatokat készítő rendszer megvalósításához, amely a távolságon és átlagos sebességen felül az út egyéb paramétereire is megbízható ajánlást tudna készíteni, akár konkrét, lokáción alapuló útvonalrajzolással együtt.